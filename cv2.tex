% LaTeX file for resume
% This file uses the resume document class (res.cls)


\documentclass{res}
%\usepackage{helvetica} % uses helvetica postscript font (download helvetica.sty)
%\usepackage{newcent}   % uses new century schoolbook postscript font
\usepackage{hyperref}

\setlength{\textheight}{9in} % increase text height to fit on 1-page

\begin{document}

\name{CHONG SHAO\\[12pt]}     % the \\[12pt] adds a blank
				        % line after name
\address{\bf E-MAIL \\ \texttt{cshao@cs.unc.edu}\\            \\ }
\address{\bf  CONTACT INFORMATION\\Campus Box 3175, Sitterson Hall\\ UNC-Chapel Hill\\ Chapel Hill, NC 27599-3175 USA\\(919) 619-0326}


\begin{resume}


\section{EDUCATION}
   \vspace{0.05in}
    University of North Carolina at Chapel Hill, Chapel Hill, NC\\
    Doctor of Philosophy, Computer Science, 2012-2017 (expected) \\
    Advisor: Stephen Pizer\\
    \\
    Polytechnic Institute of New York University,  Brooklyn, NY\\
    Bachelor of Science, Electrical Engineering and Computer Engineering, 2010-2012 \\
    G.P.A. 3.86/4.0 \\
    \\
    Courant Institute of New York University, New York, NY\\
    Graduate Level Course on Computer Vision, 2011\\
    G.P.A. 3.70/4.0\\
    \\
    Nanjing University of Posts and Telecommunications\\
    Electrical and Computer Engineering, 2007-2009\\
    G.P.A. 3.59/4.0\\

\section{EXPERIENCE}
   \vspace{-0.05in}	
   \begin{tabbing}
   \hspace{2.3in}\= \hspace{2.6in}\= \kill % set up two tab positions
    {\bf Research Assistant} \>Computer Science Department ~~~~~~~~~~~~~~  Since August 2012\\
    \> University of North Carolina at Chapel Hill  \\
                          \>Chapel Hill, NC
   \end{tabbing}\vspace{-20pt}
    Conduct research on statistics of objects in context using medial/skeletal models. Result is applied in radiology.
   \vspace{-0.1in}	
   \begin{tabbing}
   \hspace{2.3in}\= \hspace{2.6in}\= \kill % set up two tab positions
    {\bf Tutor} \>Tutoring Center ~~~~~~~~~~~~~~~~~~~~~~~~ Spring 2010 - Spring 2012\\
    \> Polytechnic Institute of NYU\\
                          \>Brooklyn, NY
   \end{tabbing}\vspace{-20pt}
    Helped students solve problems in computer science courses; made mock exams for review purpose. Worked on C++, Matlab and Python programming problems.
       \begin{tabbing}
   \hspace{2.3in}\= \hspace{2.6in}\= \kill % set up two tab positions
    {\bf Research Assistant} \> Information Systems and Internet Security Lab, Summer 2011 \\
    \>Polytechnic Institute of NYU  \\

                             \>Brooklyn, NY
   \end{tabbing}\vspace{-20pt}      % suppress blank line after tabbing
     Worked on problems in image forensics, especially on the problem of fast source camera model identification. Applied Locality-Sensitive Hashing (LSH) method to the problem, analyzed the algorithm, implemented the algorithm and analyzed the experiment result. Wrote formal reports. Gave presentations to colleagues.
   \begin{tabbing}%
   \hspace{2.3in}\= \hspace{2.6in}\= \kill % set up two tab positions
   {\bf Internship}  \>Suzhou Software Testing Center\> ~~~~~~~Summer  2008\\
                          \>Suzhou, China
   \end{tabbing}\vspace{-20pt}
    Worked with experienced engineers on several enterprise software testing tasks.   \\

\section{PROJECTS}
   \vspace{-0.05in}	
   \begin{tabbing}
   \hspace{2.3in}\= \hspace{2.6in}\= \kill % set up two tab positions
    {\bf Rablo2d} \>Research Project, UNC Chapel Hill\> ~~~~~~~~~~~~Fall 2012\\
   \end{tabbing}\vspace{-20pt}
   Rablo2d is designed to be a 2D analogy of the more-than-ten-year history software ``Pablo" in UNC-Chapel Hill medical image research group. Pablo is a tool in displaying anatomic objects in the form of skeletal models. Pablo also preforms the fitting of a skeletal model to a medical image which is used for object segmentation. In contrast, Rablo2d can also display objects in 2D in skeletal model form. And the goal of creating the 2D analogy of Pablo is to discover the multi-object relationship in terms of statistics. The software is built using the simple GUI framework ``rubyshoes". It is written in Ruby, a numerical library in Python is also used.
   \begin{tabbing}
   \hspace{2.3in}\= \hspace{2.6in}\= \kill % set up two tab positions
    {\bf Tabellae Victus} \>Undergraduate Design Project, NYU-Poly\> ~~~~~~~~~~~~Spring 2012\\
   \end{tabbing}\vspace{-20pt}
   Tabellae Victus is an implementation of the idea ``redefining document". Document should not only contain text or some figures but also more richer form of media such as video and audio. Tabellae Victus is a online document editor and viewer application. HTML5/JavaScript are used in building the front end. JSON is used in communication. Back end is built using PHP. A whole implementation of server was proposed and some initial implementation of the server was completed using C++. This project won the ``best design project" in NYU-Poly in the year of 2012. My role in the team was the developer of the front end.
  \begin{tabbing}
   \hspace{2.3in}\= \hspace{2.6in}\= \kill % set up two tab positions
    {\bf Handheld 3D Scanner} \>Computational Photography, NYU Courant\> ~~~~~~~~~~~~Fall 2011\\
   \end{tabbing}\vspace{-20pt}
    Implemented ``structure from motion" in two ways: Bundler and factorization method. Wrote code to implement factorization method. Applied two matlab toolboxes in camera calibration.
       \begin{tabbing}
   \hspace{2.3in}\= \hspace{2.6in}\= \kill % set up two tab positions
    {\bf Parallel Sorting} \>Intro to Embedded System, NYU-Poly     \>~~~~~~~~~~~Fall 2011 \\
     \end{tabbing}\vspace{-20pt}      % suppress blank line after tabbing
     Implemented parallel sorting on two Silicon Labs microcontrollers, which involved writing code for communication and sort algorithm.
   \begin{tabbing}%
   \hspace{2.3in}\= \hspace{2.6in}\= \kill % set up two tab positions
   {\bf Arithmetic Logic Unit}  \>Intro to VLSI, NYU-Poly\> ~~~~~~~~Spring 2011\\
   \end{tabbing}\vspace{-20pt}
    Designed the circuit and layout of an ALU. It contained function ADD, SUB, MUL, LSHIFT, RSHIFT, AND, OR and AND.
      \begin{tabbing}%
   \hspace{2.3in}\= \hspace{2.6in}\= \kill % set up two tab positions
   {\bf Quantum Compilers}  \>Physics of Quantum Computers, NYU-Poly\> ~~~~~~~~Spring 2011\\
   \end{tabbing}\vspace{-20pt}
    Did a survey on the proceedings of quantum compilers. Studied Dr.\ Svore's Ph.D.\ Thesis and Prof.\ Aho's research. Gave presentations to classmates and  professor.
      \begin{tabbing}%
   \hspace{2.3in}\= \hspace{2.6in}\= \kill % set up two tab positions
   {\bf Na\"{i}ve Bayes OCR}  \>SICP, UC-Berkeley\> ~~~~~~Summer 2010\\
   \end{tabbing}\vspace{-20pt}
    Implemented an optical character recognition system based on a naive Bayes classifier model in University of California, Berkeley summer session class.
    \begin{tabbing}
   \hspace{2.3in}\= \hspace{2.6in}\= \kill % set up two tab positions

    {\bf Other recent projects can be found on my GitHub page: \href{https://github.com/tingleshao}{\underline{https://github.com/tingleshao}}}\\[0.3cm]
  \end{tabbing}

\section{APPLICATIVE SKILLS}
    \vspace{0.05in}
    Proficient programming in Matlab, Python, Ruby  \\
    Familiar with C++, Scheme, Haskell, HTML/CSS, JavaScript, Java\\
    Experience in developing MVC web applications in Python and Java\\
    Familiar with Linux Administration and Programming \\
    Familiar with CMake\\
    Proficient document formatting using \LaTeX \\
    Fluent in English and Chinese

\section{SELECTED COURSES}
     \vspace{-0.05in}	
    \begin{tabbing}
    {\bf at University of North Carolina at Chapel Hill:}\\
    Medical Image Analysis, Object-Oriented Data Analysis, Scientific Computing \\
    \end{tabbing}\vspace{-30pt}
         \begin{tabbing}
    {\bf at New York University:}\\
    Computer Vision, Computational Photography, Optimization Methods\\
        \end{tabbing}\vspace{-30pt}
         \begin{tabbing}
    {\bf from Resources on the Web:}\\
    Machine Learning, Convex Optimization, Natural Language Processing
            \end{tabbing}


\section{EXTRACURRICULAR ACTIVITIES}
    \vspace{0.05in}
    Contestant in ACM-ICPC Greater NY: 2011\\
    Developer of one iPhone app on Apple app store in 2010

 \section{CERTIFICATIONS}
  \vspace{0.05in}
    Sun Certified Java Programmer (SCJP), obtained in 2008\\

\end{resume}
\end{document}
