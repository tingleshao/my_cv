% LaTeX file for resume
% This file uses the resume document class (res.cls)


\documentclass{res}
%\usepackage{helvetica} % uses helvetica postscript font (download helvetica.sty)
%\usepackage{newcent}   % uses new century schoolbook postscript font
\usepackage{hyperref}

\setlength{\textheight}{9in} % increase text height to fit on 1-page

\begin{document}

\name{CHONG SHAO\\[12pt]}     % the \\[12pt] adds a blank
				        % line after name
\address{\bf E-MAIL \\ \texttt{cshao@cs.unc.edu}\\            \\ }
\address{\bf  CONTACT INFORMATION\\Campus Box 3175, Sitterson Hall\\ UNC-Chapel Hill\\ Chapel Hill, NC 27599-3175 USA\\(919) 619-0326}


\begin{resume}


\section{EDUCATION}
   \vspace{0.05in}
    University of North Carolina at Chapel Hill, Chapel Hill, NC\\
    Doctor of Philosophy, Computer Science, 2012-2017 (expected) \\
    Advisor: Stephen Pizer\\
    \\
    Polytechnic Institute of New York University,  Brooklyn, NY\\
    Bachelor of Science, Electrical Engineering and Computer Engineering, 2010-2012 \\
    G.P.A. 3.86/4.0 \\
    \\
    Nanjing University of Posts and Telecommunications\\
    Electrical and Computer Engineering, 2007-2009\\
    G.P.A. 3.59/4.0\\

\section{EXPERIENCE}
    \vspace{-0.1in}
    \begin{tabbing}
   \hspace{2.3in}\= \hspace{2.6in}\= \kill % set up two tab positions
    {\bf Research Assistant} \>Computer Science Department ~~~~~~~~~~~~~~~~~  Since May 2013\\
    \> Visualization Group \\ 
    \> University of North Carolina at Chapel Hill  \\
                          \>Chapel Hill, NC
   \end{tabbing}\vspace{-15pt}
    1. Working on a scientifically-lossless video compression project. \\
    2. Writing programs in C++ and Matlab to perform statistical analysis on the pixels in the videos. \\
    3. The experimental videos were taken by various microscopes in the physics department. 
   \vspace{-0.1in}	
   \begin{tabbing}
   \hspace{2.3in}\= \hspace{2.6in}\= \kill % set up two tab positions
    {\bf Research Assistant} \>Computer Science Department ~~~~~~~~~~~~~~  Since August 2012\\
    \> Medical Image Analysis Group \\
    \> University of North Carolina at Chapel Hill  \\
                          \>Chapel Hill, NC
   \end{tabbing}\vspace{-15pt}
    1. Conducting research on statistics of objects in context using medial/skeletal models which has applications in radiation oncology. The experiment was done on five anatomical objects in head and neck.\\
    2. Building a prototype to define multiple sample objects in 2D, to compare the ways their spokes interact. \\
    3. Making contributions to Pablo, a large software project in UNC Medical Image group written in C++.
   \vspace{-0.1in}	
   \begin{tabbing}
   \hspace{2.3in}\= \hspace{2.6in}\= \kill % set up two tab positions
    {\bf Research Assistant} \> Information Systems and Internet Security Lab, Summer 2011 \\
    \>Polytechnic Institute of NYU  \\

                             \>Brooklyn, NY
   \end{tabbing}\vspace{-15pt}      % suppress blank line after tabbing
     Worked on problems in image forensics, especially on the problem of fast source camera model identification. Applied Locality-Sensitive Hashing (LSH) method to the problem, analyzed the algorithm, implemented the algorithm and analyzed the experiment result. Wrote formal reports. Gave presentations to colleagues.\\

\section{PROJECTS}
   \vspace{-0.015in}	
   \begin{tabbing}
   \hspace{2.3in}\= \hspace{2.6in}\= \kill % set up two tab positions
    {\bf Rablo2d} \>Research Project, UNC Chapel Hill\> ~~~~~~~~~~~Fall 2012\\
   \end{tabbing}\vspace{-20pt}
   Rablo2d is designed to be a 2D analogy to ``Pablo", a software from UNC-Chapel Hill medical image research group. Pablo is a tool in displaying anatomic objects in the form of skeletal models. Pablo also performs the fitting of a skeletal model to a medical image which is used for object segmentation. In contrast, Rablo2d can also display objects in 2D in skeletal model form. And the goal of creating the 2D analogy of Pablo is to discover the multi-object relationship in terms of statistics. The software is built using the simple GUI framework ``rubyshoes". It is written in Ruby; a numerical library in Python is also used.
   \begin{tabbing}
   \hspace{2.3in}\= \hspace{2.6in}\= \kill % set up two tab positions
    {\bf Tabellae Victus} \>Undergraduate Design Project, NYU-Poly\> ~~~~~~~~Spring 2012\\
   \end{tabbing}\vspace{-20pt}
   Tabellae Victus is an implementation of the idea ``redefining document". A document should not only contain text or some figures but also more richer form of media such as video and audio. Tabellae Victus is a online document editor and viewer application. HTML5/JavaScript are used in building the front end. JSON is used in communication. The back end is built using PHP. A whole implementation of server was proposed and some initial implementation of the server was completed using C++. This project won the ``best design project" in NYU-Poly in the year of 2012. My role in the team was the developer of the front end.
  \begin{tabbing}
   \hspace{2.3in}\= \hspace{2.6in}\= \kill % set up two tab positions
    {\bf Handheld 3D Scanner} \>Computational Photography, NYU Courant\> ~~~~~~~~~~~~Fall 2011\\
   \end{tabbing}\vspace{-20pt}
    Implemented ``structure from motion" in two ways: Bundler and factorization method. Wrote code to implement factorization method. Applied two matlab toolboxes in camera
  \begin{tabbing}
     \hspace{2.3in}\= \hspace{2.6in}\= \kill % set up two tab positions

    {\bf GitHub page: \href{https://github.com/tingleshao}{\underline{https://github.com/tingleshao}}}\\[0.3cm]
  \end{tabbing}

\section{APPLICATIVE SKILLS}
    \vspace{0.05in}
    Proficient programming in Matlab, Python, Ruby  \\
    Familiar with C++, Scheme, Haskell, JavaScript, Java\\
    Familiar with Linux Administration and Programming \\
    Familiar with Make, CMake, QMake\\
    Proficient document formatting using \LaTeX \\

\section{SELECTED COURSES}
     \vspace{-0.05in}	
    \begin{tabbing}
    {\bf at the University of North Carolina at Chapel Hill:}\\
    Parallel and Distributed Computing\\
    Medical Image Analysis, Object-Oriented Data Analysis, Scientific Computing \\
    Visual Solid Shape, Numerical Linear Algebra, Programming Language Concepts\\
    \end{tabbing}\vspace{-30pt}
         \begin{tabbing}
    {\bf at New York University:}\\
    Computer Vision, Computational Photography, Optimization Methods\\
        \end{tabbing}\vspace{-30pt}
         \begin{tabbing}
    {\bf from Resources on the Web:}\\
    Machine Learning, Convex Optimization, Natural Language Processing
            \end{tabbing}


\section{EXTRACURRICULAR ACTIVITIES}
    \vspace{0.05in}
    Contestant in ACM-ICPC Greater NY: 2011\\

\end{resume}
\end{document}
